\documentclass[12pt,a4paper]{article}
\usepackage[utf8]{inputenc}
\usepackage{amsmath}
\usepackage{amsfonts}
\usepackage{amssymb}
\usepackage[english]{babel}
\usepackage[left=2cm,right=2cm,top=2cm,bottom=2cm]{geometry}
\begin{document}
\begin{center}
\section*{Sample Proposal format}
Lab name, Address, City, State, Zip Code
\end{center}
{\bf Proposal submitted to: National Science Foundation, Box 000, Washington, DC 00000}
\subsection*{Principle Investigators' names:}
\section*{Sample Proposal format}
\subsection*{Introduction}
\begin{itemize}
\item Indicate the purpose of the proposal. What are you proposing to do?
\item Indicate the rational behind your proposal. Why is this necessary or important?
\end{itemize}
\subsection*{Preliminary Analysis of a Subset of Samples}
\begin{itemize}
\item Identify all specimens tested (include all specimens tested by groups in your lab).
\item Summarize the results of the preliminary testing of {\bf all samples} examined in your lab. What do the results indicate? For example, can you use the results to estimate the percentage of the samples that are alive vs. dead, organic vs. inorganic?
\end{itemize}
\subsection*{Proposal for Analysis of the remaining 8,000 samples}
\begin{itemize}
\subsubsection*{Materials and Methods}
\item Indicate the operational definitions of you plan to use for alive versus dead, etc., and what, if any, assumptions you plan to make in conducting the analysis
\item Provide a general summary description of the test{s} you plan to perform to determine if the specimens are dead or alive, etc. This text must be accompanied by a finalized procedural flow diagram.  How have you improved your experimental design for the analysis of the remaining samples based on the preliminary analysis of your class data?
\item Include a description of any controls needed, as well as, when and why they will be needed.%
\subsubsection*{Results}%
\item Indicate how you will analyze the results of the procedures proposed.%
\subsubsection*{Discussion}%
\item Indicate any recommendations you might make for future study of the specimens based on analysis of the preliminary results obtained. How does this relate to the propse and rationale presented in your introduction?
\end{itemize}
\subsection*{Budget}
\begin{itemize}
\item Detail the costs associated with the protocol proposed. Formally state any assumptions you made in making the estimates. Using these estimates, indicate what it would cost the government for your lab to do the preliminary analysis of 8,000 samples.
\end{itemize}
\subsection*{References}
Include the complete citations here for all references cited in the text of the proposal.
\subsection*{Tables and figures}
Group all tables and figures after the references section.
\subsection*{Appendix}
The appendix should contain the following in this order:
\begin{itemize}
\item Your original week 1 proposal with all review comments and revisions.
\item Your preliminary rough drafts of final proposal sections submitted at the beginning of lab week 3.
\item Your copy of the spreadsheet showing all class data. Be sure to highlight your group's data on this sheet.
\end{itemize}
\end{document}